\documentclass[12pt]{article}

\usepackage{amssymb}
\usepackage[top=5em, bottom=5em, left=5em, right=5em]{geometry}
\usepackage{listings}
\usepackage{tikz}
\usetikzlibrary{positioning}

\setlength\parindent{0em}
\setlength\parskip{1em}

\title {Assignment 5}

\author {Hendrik Werner s4549775}

\begin{document}
\maketitle

This was done in collaboration with Constantin Blach (s4329872).

\section{} %1
\section{} %2
\section{} %3
\section{} %4
We can treat the currencies $c_1, c_2, \dots, c_n$ as vertices and the exchange rates $r_{i, j}$ as directed edges $(c_i, c_j)$ with weight $r_{i, j}$. $G = (V, E), V = c_1, c_2, \dots, c_n, E = \{r_{i, j}\ |\ i, j \in \mathbb{N^+}, i \leq n, j \leq n\}$

We want to find the optimal path from $s \in V$ to $t \in V$. The optimal path is defined as the path that maximises the output of $t$. This is the path with the highest accumulative exchange rate. There may not exists an optimal path if $G$ contains circles with an accumulative exchange rate greater than $1$.

Exchange rates are accumulated by multiplication.

We can find the optimal path by modifying the Bellman-Ford algorithm:

\begin{lstlisting}
d[s] <- 0
for v in V - s
	d[v] <- infinity

for i in 1..|V|-1
	for u, v in E
		if d[v] < d[u] * w(u, v)
			d[v] <- d[u] * w(u, v)

for u, v in E
	if d[v] < d[u] * w(u, v)
		throw MoneyPrintingMachineException
\end{lstlisting}

Where $infinity = \infty$ and $w(u, v) = r_{u, v}$ and $MoneyPrintingMachineException$ indicates that we found a loop with an accumulative exchange rate greater than $1$ which means we can make money by taking this loop so there is no optimal path as we can always increase the accumulative exchange rate by taking another round trip in this loop.

\section{} %5

\end{document}
